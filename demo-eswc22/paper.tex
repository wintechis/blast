% This is samplepaper.tex, a sample chapter demonstrating the
% LLNCS macro package for Springer Computer Science proceedings;
% Version 2.20 of 2017/10/04
%
\documentclass[runningheads]{llncs}
%
%\usepackage{fullpage}
%\usepackage{setspace}
%\onehalfspacing
\usepackage{listings}
\usepackage{graphicx}
% Used for displaying a sample figure. If possible, figure files should
% be included in EPS format.
\usepackage{makecell}
\renewcommand\theadfont{\bfseries}
%
% If you use the hyperref package, please uncomment the following line
% to display URLs in blue roman font according to Springer's eBook style:
%\renewcommand\UrlFont{\color{blue}\rmfamily}

\begin{document}
%
\title{BLAST: Block Applications for Things}
%
%\titlerunning{Abbreviated paper title}
% If the paper title is too long for the running head, you can set
% an abbreviated paper title here
%
\author{Michael Freund\inst{1}\orcidID{0000-0003-1601-9331} \and
Thomas Wehr\inst{1}\orcidID{0000-0002-0678-5019} \and
Andreas Harth\inst{1,2}\orcidID{0000-0002-0702-510X}}
%
\authorrunning{M. Freund et al.}
% First names are abbreviated in the running head.
% If there are more than two authors, 'et al.' is used.
%
\institute{Fraunhofer Institute for Integrated Circuits IIS, Nürnberg, Germany 
\email{firstname.lastname@iis.fraunhofer.de}\\
\and
Friedrich–Alexander University Erlangen-Nürnberg, Nürnberg, Germany}
%
\maketitle              % typeset the header of the contribution
%
\begin{abstract}
  We introduce a block-based visual programming language called BLAST for programs involving connected devices with a Web of Things abstraction.
  We developed an editor and an execution environment for BLAST programs that runs in a web browser.
  We demonstrate that BLAST can be used to create programs that interact with a variety of devices.
  In particular we show the use of connected devices in a geofencing scenario.
\keywords{Block-based Programming \and Web of Things.}
\end{abstract}
%
%
%
\setcounter{footnote}{0}
\section{Introduction}
Connected devices form the basis for a wide variety of applications in industry settings, such as condition monitoring, energy consumption analysis, simulation and optimisation~\cite{CIMINO2019103130}.
%Given the multitude of protocols for interacting with such devices, application development is difficult.
The Web of Things (WoT)\footnote{\url{https://www.w3.org/WoT/}} aims at providing a set of standardised technologies to help simplify creating applications involving connected devices.
At the same time, casual users should be able to quickly program applications involving connected devices.
The programming languages currently in use for accessing devices are primarily text-based and are difficult to access for casual users.
One possible approach to improve accessibility and ease the creation of computer programs is the visual programming paradigm.

Visual programming does not require the user to enter any text; instead, programs are based on graphical elements like blocks, graphs, tables or diagrams.
Programs are created by arranging predefined blocks with images or text via a drag-and-drop interface.
Blocks can be geometrically aligned with other compatible blocks, like in a jigsaw puzzle, to form complex programs.
Since blocks can only be arranged in the correct way, no syntax errors can occur~\cite{10.11453341221}~\cite{maloney2010scratch}~\cite{10.1145/1089733.1089734}.

The created block programs can be translated into any predefined programming language, resulting in a normal source code. 
The block-based approach can be used not only for learning programming concepts, but also for more complex problems.
For example, block languages have been successfully used in constructing SPARQL queries~\cite{7369012} or programming industrial robots~\cite{8120406}~\cite{tomlein2017visual}, both areas which normally require advanced programming skills.

We present BLAST, a visual programming environment that works with devices following the Web of Things interface.
BLAST programs can be created in a web browser, based on Google's Blockly\footnote{\url{https://developers.google.com/blockly/}}, a JavaScript library for creating block-based languages and editors.
BLAST programs can be translated into JavaScript and executed in the web browser.
BLAST implements the Web of Things abstraction to devices with Bluetooth Low Energy (BLE) and USB Human Interface Devices (HID) interface via APIs of modern web browsers.

The currently best known tool for programming IoT devices using visual programming is Node-RED\footnote{\url{https://nodered.org/}}.
Node-RED's programming paradigm is based on dataflows; in contrast, BLAST focuses on control flow constructs.
The use cases in industrial settings we envision for BLAST involve mostly control flow, and in such scenarios a dataflow abstraction can lead to programs that are difficult to understand and maintain.
%The nodes in these graphs are black box processes, created by the developers, that can be connected to each other by the end user.
%The block-based approach utilised by BLAST provides more flexibility as its blocks are a lot more fine grained.
%BLAST is a programming language with a

Another system similar to BLAST is Punya~\cite{patton2021punya}, an Android app development system based on the MIT App Inventor\footnote{\url{https://appinventor.mit.edu/}}.
Like BLAST, Punya offers a block-based programming environment for writing programs that can include SPARQL queries and access online services.
Punya focuses on creating applications for Android devices that can access the mobile phone's internal hardware.
Punya also acts as a client to publish and access data on servers supporting the Constrained Application Protocol (CoAP).
BLAST programs, on the other hand, run in modern web browsers and support devices based on a WoT abstraction.

In the following we first introduce the BLAST language using a geofencing scenario, next describe the browser-based editor and execution environment for BLAST programs and then conclude.

\section{BLAST Language}

When choosing the vocabulary for our block-based language BLAST, we decided on a mixture of natural language and computer language~\cite{8120404}.
In BLAST, the blocks are arranged into readable sentences, as in a natural language, but the blocks also contain a few technical programming terms, such as ''repeat while true'' instead of ''forever''.

The BLAST language includes standard elements from other programming languages such as loops, functions, and variables.
In addition, the BLAST language provides blocks related to interacting with WoT devices.
These WoT-related blocks can be grouped under the three interaction affordances Properties (that can be read and written), Actions (that can be invoked), and Events (that can be observed).
Further, the BLAST language includes blocks to send HTTP requests with arbitrary headers, request method and request body.
The BLAST language also includes blocks to load RDF graphs from a URI and execute SPARQL queries, which allows for the interaction with Read-Write Linked Data APIs or other REST APIs.
Finally, if the available blocks are not sufficient to implement a desired functionality, there is a block in which native JavaScript code can be entered, and thus almost everything that is possible in JavaScript is also possible in BLAST.

For an example of a BLAST program see figure \ref{fig1}.
The program involves a LED light with a BLE interface and a small battery-powered BLE beacon.
The BLE controller (the ``BLE central'') is implicitly used when reading the signal-strength property.
The functionality of these devices (``Things'') can be accessed via reading and writing properties.
The LED light, for example, has a property ``colour'' that can be written.
To enable an event abstraction, the program includes a block that defines a state with a condition on the signal strength of the BLE beacon.
On entering or exiting the defined state, the event blocks, which can contain arbitrary blocks, are triggered.

\begin{figure}
\includegraphics[width=\textwidth]{screenshot 3.png}%im Ordner sind noch 2 alternative Screenshots
\caption{Example BLAST program that implements a geofencing scenario: when the beacon gets close, the light is turned off; when the beacon leaves the close range, a sound is played and the light is turned red.} \label{fig1}
\end{figure}

\section{BLAST Editor and Execution Environment}

We have implemented an editor and an execution environment for BLAST programs that run in modern web browsers.

The editor uses the Blockly library and provides custom control and command blocks that make up the BLAST language.
Based on the user's arrangement of these blocks, the editor generates JavaScript code that can then be executed in the browser.
The system builds on the Web Bluetooth API\footnote{\url{https://webbluetoothcg.github.io/web-bluetooth/}} and the WebHID API\footnote{\url{https://wicg.github.io/webhid/}} to interact with devices.
The system uses the Web of Things abstraction as interface; we have implemented the WoT interface for eight devices that use the BLE GAP and GATT profiles as well as the USB Human Interface Devices (HID) interface.
The WoT interface contains the metadata and interaction affordances as well as the JavaScript code needed for communication.
In addition, the system includes access to the camera and sound output via browser APIs using the WoT interface.
Further, the system includes access to the Web Speech API\footnote{\url{https://wicg.github.io/speech-api/}}.
The functionality to issue HTTP requests is implemented using the fetch API~\footnote{\url{https://developer.mozilla.org/en-US/docs/Web/API/Fetch\_API/Using\_Fetch}}, Social Linked Data (Solid) authentication is implemented using the inrupt solid library~\footnote{\url{https://docs.inrupt.com/developer-tools/javascript/client-libraries/}}, and SPARQL query support is implemented using µRDF.js\footnote{\url{https://github.com/vcharpenay/uRDF.js/}}.
Next to the presented small scenario around geofences, we have also developed BLAST programs to control and monitor an automated picking line to prepare orders in a more complex logistics scenario.

\section{Conclusion}

We have presented BLAST, a block-based language that simplifies the interaction with IoT devices, digital twins or other web resources without the need for a deeper understanding of different programming languages or network protocols.
We are currently working on an execution environment that runs on Node.js as a way to execute BLAST programs independently of browsers. This will allow BLAST to use the full Bluetooth specification and not only features exposed by the Web Bluetooth API, which restricts the set of accessible GATT attributes in a blocklist~\footnote{\url{https://webbluetoothcg.github.io/web-bluetooth/\#the-gatt-blocklist}} to provide security and privacy protection.
%To this end, we are developing an execution environment that can load BLAST programs, convert them to JavaScript, and execute them as a server.
%In the future, it would enrich BLAST if there was a way to consume WoT TDs and derive interaction possibilities, which would allow BLAST to interact with any WoT device.
%@@@Web Bluetooth has certain restrictions (security, reading RSSI), server should not have these restrictions.

%
% ---- Bibliography ----
%
% BibTeX users should specify bibliography style 'splncs04'.
% References will then be sorted and formatted in the correct style.
%
\bibliographystyle{splncs04}
\bibliography{references}
%

\end{document}
